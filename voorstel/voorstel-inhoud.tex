%---------- Inleiding ---------------------------------------------------------

\section{Introductie}%
\label{sec:introductie}

Mainframe is een van de oudste en meest gebruikte computertechnologieën ooit, maar na al die jaren dat er getwijfeld werd over het wel of niet afschaffen ervan. 
Is er een groot tekort aan pas afgestudeerden die het voortouw zouden kunnen nemen in de vernieuwing van de mainframe. 
Innovaties zoals IBM Dependency Based Build zorgen ervoor dat het platform moderner is dan ooit door de welgekende DevOps workflow mogelijk te maken voor een mainframe applicatie. 
Deze workflow bestaat uit een pipeline en een repository, in het geval van dit onderzoek zal de pipeline(s) verzorgd worden door Azure Pipelines en de repositories zullen aangeboden worden via Azure Repos. 
Beide services maken deel uit van het Azure DevOps pakket dat nog heel vaak zal vermeld worden doorheen het onderzoek en de proof of concept.
\\ \\
De leveranciers van mainframe software hebben het belang van modernisering in het landschap opgemerkt en zijn dus al enkele jaren volop tijd en geld aan het pompen in nieuwe gebruiksvriendelijke en meer hedendaagse tools.
Die tools zouden het werken op zo'n omgeving vereenvoudigen voor zowel gebruikers, ontwikkelaars en administrators. 
\\ \\
Het doel van dit onderzoek is dan ook om een coolgen programma, gemaakt met een moderne Git ondersteunende Integrated Development Environment, kortweg IDE, zoals bijvoorbeeld Visual Studio Code.
Volledig automatisch te kunnen compileren en binden op het moment dat er wijzigingen opgeslagen worden naar de Azure Repo dit zal door een Azure Pipeline in gang gezet worden.
Hierdoor wordt het bestaande proces dat een aanzienlijk aantal stappen extra heeft vereenvoudigd tot één stap, namelijk de applicatie opslaan en pushen naar Azure Repo. 
\\ \\
Het onderzoek zal als een succes worden beschouwd indien de proof of concept een positief resultaat geeft, dit wil zeggen dat er moet aangetoond worden dat het mogelijk is om een coolgen programma 
automatisch te laten compileren en binden door een Azure Pipeline met behulp van IBM DBB. 
Verder moet ook aangetoond worden dat er versiebeheer mogelijk is met behulp van Azure Repos om zo naar verschillende versies van programma's terug te kunnen keren. 
Indien beide vereisten worden ingelost kan er gesproken worden van een succesvol onderzoek en bijgevolg een succesvolle proof of concept. 

%---------- Stand van zaken ---------------------------------------------------

\section{State-of-the-art}%
\label{sec:state-of-the-art}

Hier beschrijf je de \emph{state-of-the-art} rondom je gekozen onderzoeksdomein, d.w.z.\ een inleidende, doorlopende tekst over het onderzoeksdomein van je bachelorproef. Je steunt daarbij heel sterk op de professionele \emph{vakliteratuur}, en niet zozeer op populariserende teksten voor een breed publiek. Wat is de huidige stand van zaken in dit domein, en wat zijn nog eventuele open vragen (die misschien de aanleiding waren tot je onderzoeksvraag!)?

Je mag de titel van deze sectie ook aanpassen (literatuurstudie, stand van zaken, enz.). Zijn er al gelijkaardige onderzoeken gevoerd? Wat concluderen ze? Wat is het verschil met jouw onderzoek?

Verwijs bij elke introductie van een term of bewering over het domein naar de vakliteratuur, bijvoorbeeld~\autocite{Hykes2013}! Denk zeker goed na welke werken je refereert en waarom.

Draag zorg voor correcte literatuurverwijzingen! Een bronvermelding hoort thuis \emph{binnen} de zin waar je je op die bron baseert, dus niet er buiten! Maak meteen een verwijzing als je gebruik maakt van een bron. Doe dit dus \emph{niet} aan het einde van een lange paragraaf. Baseer nooit teveel aansluitende tekst op eenzelfde bron.

Als je informatie over bronnen verzamelt in JabRef, zorg er dan voor dat alle nodige info aanwezig is om de bron terug te vinden (zoals uitvoerig besproken in de lessen Research Methods).

% Voor literatuurverwijzingen zijn er twee belangrijke commando's:
% \autocite{KEY} => (Auteur, jaartal) Gebruik dit als de naam van de auteur
%   geen onderdeel is van de zin.
% \textcite{KEY} => Auteur (jaartal)  Gebruik dit als de auteursnaam wel een
%   functie heeft in de zin (bv. ``Uit onderzoek door Doll & Hill (1954) bleek
%   ...'')

Je mag deze sectie nog verder onderverdelen in subsecties als dit de structuur van de tekst kan verduidelijken.

%---------- Methodologie ------------------------------------------------------
\section{Methodologie}%
\label{sec:methodologie}

Hier beschrijf je hoe je van plan bent het onderzoek te voeren. Welke onderzoekstechniek ga je toepassen om elk van je onderzoeksvragen te beantwoorden? Gebruik je hiervoor literatuurstudie, interviews met belanghebbenden (bv.~voor requirements-analyse), experimenten, simulaties, vergelijkende studie, risico-analyse, PoC, \ldots?

Valt je onderwerp onder één van de typische soorten bachelorproeven die besproken zijn in de lessen Research Methods (bv.\ vergelijkende studie of risico-analyse)? Zorg er dan ook voor dat we duidelijk de verschillende stappen terug vinden die we verwachten in dit soort onderzoek!

Vermijd onderzoekstechnieken die geen objectieve, meetbare resultaten kunnen opleveren. Enquêtes, bijvoorbeeld, zijn voor een bachelorproef informatica meestal \textbf{niet geschikt}. De antwoorden zijn eerder meningen dan feiten en in de praktijk blijkt het ook bijzonder moeilijk om voldoende respondenten te vinden. Studenten die een enquête willen voeren, hebben meestal ook geen goede definitie van de populatie, waardoor ook niet kan aangetoond worden dat eventuele resultaten representatief zijn.

Uit dit onderdeel moet duidelijk naar voor komen dat je bachelorproef ook technisch voldoen\-de diepgang zal bevatten. Het zou niet kloppen als een bachelorproef informatica ook door bv.\ een student marketing zou kunnen uitgevoerd worden.

Je beschrijft ook al welke tools (hardware, software, diensten, \ldots) je denkt hiervoor te gebruiken of te ontwikkelen.

Probeer ook een tijdschatting te maken. Hoe lang zal je met elke fase van je onderzoek bezig zijn en wat zijn de concrete \emph{deliverables} in elke fase?

%---------- Verwachte resultaten ----------------------------------------------
\section{Verwacht resultaat, conclusie}%
\label{sec:verwachte_resultaten}

Hier beschrijf je welke resultaten je verwacht. Als je metingen en simulaties uitvoert, kan je hier al mock-ups maken van de grafieken samen met de verwachte conclusies. Benoem zeker al je assen en de onderdelen van de grafiek die je gaat gebruiken. Dit zorgt ervoor dat je concreet weet welk soort data je moet verzamelen en hoe je die moet meten.

Wat heeft de doelgroep van je onderzoek aan het resultaat? Op welke manier zorgt jouw bachelorproef voor een meerwaarde?

Hier beschrijf je wat je verwacht uit je onderzoek, met de motivatie waarom. Het is \textbf{niet} erg indien uit je onderzoek andere resultaten en conclusies vloeien dan dat je hier beschrijft: het is dan juist interessant om te onderzoeken waarom jouw hypothesen niet overeenkomen met de resultaten.

