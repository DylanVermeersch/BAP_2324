%%=============================================================================
%% PROOF-OF-CONCEPT
%%=============================================================================

\chapter{\IfLanguageName{dutch}{Werking huidig systeem}{Operation of the current system}}%
\label{ch:current system)}

\section{Inleiding}
\label{sec:inleiding}
Dit hoofdstuk zal de werking van het huidige systeem om coolgen applicaties te ontwikkelen van ArcelorMittal Gent toelichten en visualiseren. De stappen die in het huidige systeem aanwezig zijn zullen ook terug te vinden zijn in het nieuwe systeem met behulp van een pipeline en IBM DBB. Het is uiterst noodzakelijk om kennis te hebben van het huidige systeem alvorens een nieuw systeem kan gemaakt/geïmplementeerd worden. Dit komt doordat het systeem dat momenteel gebruikt uitermate veel is gepersonaliseerd door ArcelorMittal Gent om aan hun eisen en gebruiken te voldoen. 

\section{Compileren}
\label{sec:compileren}
De huidige geïnstalleerde versie van de Cobol compiler binnen ArcelorMittal Gent is versie 6.20, deze versie van Cobol is geïntroduceerd in 2017 op 8 september en heeft geen support meer vanaf 30 september 2024.
\\ \\
De JCL die uitgevoerd wordt wanneer men een compilatie van een programma wil uitvoeren wordt opgeroepen via een product van Rocket software namelijk MSP (Manager Products). Dit product zorgt ervoor dat een ontwikkelaar kan meegeven welk programma hij/zij wil compileren en in welke omgeving die dat wil doen, dit kan bijvoorbeeld productie of ontwikkeling zijn. Dat product zorgt er dan voor dat de JCL voor de compile uit te voeren wordt gestart met de juiste parameters en de juiste libraries die moeten meegegeven worden tijdens de compilatie zoals de SYSLIB en SYSIN DD's.
\\ \\
Heel belangrijk binnen het huidig systeem is het gebruik van meta-data, hierdoor weet in dit geval MSP welke parameters die moet aanvoeren aan de compiler, welke libraries die moet alloceren en of er extra zaken moeten gestart worden zoals een Db2 package bind of Db2 plan bind. Deze meta-data is dan ook een van de eerste zaken dat gecontroleerd wordt als men een compilatie wil starten. Indien er geen meta-data te vinden is dan zal de compile geannuleerd worden. 
\\ \\
Deze meta-data bevat onder andere informatie over de afdeling waar het gemaakt is, de programmeur van de applicatie, wat voor soort applicatie het is. Zo heb je 3 verschillende soorten programma's binnen ArcelorMittal Gent, je hebt de main programma's, die kunnen ofwel volledig onafhankelijk draaien of ze roepen sub programma's op. Deze main applicaties kunnen zelf niet opgeroepen worden door andere applicaties. Als laatste zijn er ook nog 2 soorten sub programma's, de fsub en de sub. In theorie is er niet veel verschil buiten de manier waarop ze opgeroepen kunnen worden. Zo wordt een sub statisch gebindt aan een programma dat hem oproept en een fsub wordt dynamisch opgeroepen tijdens de run time. Naast die 3 soorten van applicaties is er ook nog het feit of er gebruik gemaakt wordt van subsystemen zoals IMS, Db2 of MQ. Indien hiervan gebruik gemaakt wordt dan kan de ontwikkelaar ook deze zaken aanduiden binnen het meta-data scherm van zijn/haar applicatie. 
\\ \\
Eenmaal alle meta-data aanwezig is zal MSP met behulp van skeletons de juiste JCL opmaken om die applicatie te compileren met de juiste libraries en parameters. De compile parameters zijn heel gelijkaardig voor alle soorten programma's maar kunnen toch nog ergens licht afwijken. Zoals wanneer er gebruik gemaakt wordt van een subsysteem zoals IMS of Db2.
\\ \\

\section{Bind}
\label{sec:bind}
Er wordt gebruik gemaakt van de IEWL Binder voor z/OS 2.5 om de applicaties van ArcelorMittal te binden. Het belangrijkste verschil tussen een programma dat gebind is met IEWL en een dat gelinkedit is, is het feit dat bij de linkedit een groot uitvoerbaar bestand gemaakt wordt van de verschillende objectbestanden van het hoofdprogramma, de subprogramma's en de subroutines die opgeroepen worden. Hierdoor is het programma dat gelinkedit wordt statisch aangemaakt, dit wil zeggen dat indien er een sub programma verandert er dus een nieuwe linkedit moet gebeuren van alle progamma's die dat sub programma gebruiken. Met een binder kan je dynamisch een programma aan een ander programma koppelen/binden, op die manier is het zo dat een sub programma opgeroepen wordt op run time en niet tijdens de compile. Hierdoor hoeft er geen herlink meer te gebeuren van de applicaties die dat programma gebruiken. 
\\ \\
Er zijn net zoals bij de compilatie ook parameters die meegegeven worden aan de binder om op de juiste manier de programma's te kunnen binden. In tegenstelling als bij de compilatie moet er niet voor elk soort programma (main, Db2, IMS, sub, ...) een aparte parameter lijst gemaakt worden. De parameters zijn hetzelfde voor main- en fsub programma's aangezien die worden opgeslagen als DLL, voor de sub programma's is er een andere parameter lijst die ervoor zorgt dat de sub geen DLL wordt maar statisch blijft. Hierdoor zullen main- en fsub programma's wel dynamisch opgeroepen kunnen worden tijdens run time en zal er voor sub programma's opnieuw met herlink en hercompile moeten gewerkt worden. 
\\ \\ 
De binder wordt net zoals de compiler opgeroepen door het product MSP, het gaat op dezelfde manier te werk als bij de compile en maakt gebruik van een applicatie zijn meta-data om zo de juiste libraries en parameters mee te geven.







