%%=============================================================================
%% Voorwoord
%%=============================================================================

\chapter*{\IfLanguageName{dutch}{Woord vooraf}{Preface}}%
\label{ch:voorwoord}

Voor u ligt mijn bachelorproef getiteld \enquote{Onderzoek naar de compilatie en bind van een Coolgen programma via een Azure DevOps pipeline binnen de mainframe omgeving van ArcelorMittal Gent met proof of concept}. 
Deze bachelorproef is geschreven als een vereiste voor het behalen van mijn afstudeerdiploma in de richting \enquote{Toegepaste Informatica} aan HoGent.
Gedurende de periode van februari tot mei 2024 heb ik me beziggehouden met het onderzoeken en het schrijven van deze bachelorproef. 
\\ \\
Gedurende mijn tijd aan HoGent heb ik lang getwijfeld waar ik later mijn job van wou maken, er was niets dat mijn interesse voor de volle 100\% kon wekken. 
Dit allemaal veranderde toen er op het einde van mijn tweede jaar in de opleiding Toegepaste Informatica de term mainframe naar mijn hoofd werd geslingerd. 
Ik was meteen heel erg geïnteresseerd in deze nieuwe term en vroeg mijn medestudenten Muhammed en Mehmet om meer informatie over het onderwerp. 
Het bleek al snel dat het een hot topic was binnen de wereld van big IT, we hebben toen ook de vraag gesteld aan Dhr. Leendert Blondeel of het mogelijk was om meer duiding te geven bij de specialisatierichting Mainframe Expert. 
Dit verzoek werd met veel enthousiasme onthaald en vrijwel meteen werd een virtueel gesprek ingepland met mijzelf, Mehmet en Muhammed om ons meer te vertellen over mainframe en hoe een jaar van een mainframe student op HoGent eruitziet.
Meteen tijdens het gesprek wist ik dat ik mijn afstudeerrichting gevonden had, de passie was en is nog altijd heel erg groot bij Dhr. Leendert Blondeel en trok je direct aan zoals normaal alleen een magneet dat kan. 
\\ \\
Het werd tijdens dat derde jaar alleen maar duidelijker dat ik de juiste weg ingeslagen was, we ontmoetten enorm veel mensen die net zoals Dhr. Leendert Blondeel heel erg gepassioneerd zijn over mainframe en alles daaromtrent. 
Veel van deze mensen zijn nog altijd goed bereikbaar via mail ondanks het feit dat ze bij heel grote bedrijven zitten, kan ik ze nog altijd bereiken voor allerhande vragen over bijeenkomsten, mainframe topics of zelfs om de ondervindingen van mijn bachelorproef mee te delen. 
Het heeft me duidelijk geen windeieren gelegd om de keuze voor mainframe te maken. 
\\ \\
Graag wil ik mijn begeleider, Dhr. Leendert Blondeel, bedanken voor zijn uitstekende begeleiding en ondersteuning gedurende dit onderzoek. 
Zijn begeleiding heeft mijn leermogelijkheden gemaximaliseerd, waarvoor ik hem zeer dankbaar ben. 
Ook wil ik Dhr. Didier Marichal van ArcelorMittal Gent bedanken voor zijn bijdrage aan dit onderzoek als co-promotor. 
\\ \\
Tot slot wil ik ook het mainframe systeembeheer team van ArcelorMittal Gent bedanken, ik kon elk moment van de dag bij hun terecht met allerhande vragen of zelfs om een gezellig babbeltje te slaan. 
Ze zorgden ervoor dat ik me heel erg thuis voelde binnen het team. 
Graag wil ik ook nog mijn familie en vrienden bedanken voor hun steun gedurende mijn onderzoeksproces. 
\\ \\
Ik wens u veel leesplezier.
\\ \\
Dylan Vermeersch
\\
Oostakker, 24 mei 2024