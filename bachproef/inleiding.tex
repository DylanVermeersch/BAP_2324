%%=============================================================================
%% Inleiding
%%=============================================================================

\chapter{\IfLanguageName{dutch}{Inleiding}{Introduction}}%
\label{ch:inleiding}

Deze bachelorproef gaat over op welke manier er een Azure DevOps pipeline kan geïntegreerd worden binnen ArcelorMittal Gent om hun bestaande en toekomstige Coolgen programma's te compilen/binden. 
Dit onderzoek is begonnen omdat ArcelorMittal Gent graag hun mainframe omgeving wil moderniseren en automatiseren, meer bepaald gaat het in dit onderzoek over het automatiseren van de compile/bind van Coolgen programma's.



\section{\IfLanguageName{dutch}{Probleemstelling}{Problem Statement}}%
\label{sec:probleemstelling}

Bij ArcelorMittal Gent wordt er gekeken hoe men hun mainframe omgeving kan moderniseren, op die manier kunnen ze efficiënt gebruikmaken van de nieuwe technologieën die de laatste jaren in opmars zijn zoals automatische DevOps pipelines. 
Door zaken zoals het compileren/binden van programma's automatisch te starten via een pipeline wordt er aan tijd gewonnen wat er dus voor zorgt dat een taak efficiënter wordt. 
Bij ArcelorMittal Gent zijn ze daarom opzoek naar een manier om het compile/bind proces voor hun Coolgen programma's te automatiseren met behulp van een Azure DevOps pipeline.
Momenteel is het nog niet duidelijk wat ervoor nodig is om zo'n Azure DevOps pipeline werkende te krijgen voor bestaande en toekomstige Coolgen programma's binnen ArcelorMittal Gent.

\section{\IfLanguageName{dutch}{Onderzoeksvraag}{Research question}}%
\label{sec:onderzoeksvraag}

Op welke manier kan er een Azure DevOps pipeline geïntegreerd worden om de compile/bind van bestaande en toekomstige Coolgen programma's uit te voeren. 
Wat zijn de gevolgen voor het huidige compile/bind systeem voor Coolgen programma's na het integreren van de bekomen pipeline met de mainframe omgeving van ArcelorMittal Gent. 

\section{\IfLanguageName{dutch}{Onderzoeksdoelstelling}{Research objective}}%
\label{sec:onderzoeksdoelstelling}

Het onderzoek moet een duidelijk beeld geven over welke technologieën er gebruikt worden om een Azure DevOps pipeline op te zetten die als taak heeft om de compile/bind van Coolgen programma's te verzorgen.
Een proof of concept wordt opgesteld om zo'n Azure DevOps pipeline uit te werken die de compile/bind van bestaande en toekomstige Coolgen programma's kan uitvoeren. 

\section{\IfLanguageName{dutch}{Opzet van deze bachelorproef}{Structure of this bachelor thesis}}%
\label{sec:opzet-bachelorproef}
De rest van deze bachelorproef is als volgt opgebouwd:

In Hoofdstuk~\ref{ch:stand-van-zaken} wordt een overzicht gegeven van de stand van zaken binnen het onderzoeksdomein, op basis van een literatuurstudie.

In Hoofdstuk~\ref{ch:methodologie} wordt de methodologie toegelicht en worden de gebruikte onderzoekstechnieken besproken om een antwoord te kunnen formuleren op de onderzoeksvragen.

In Hoofdstuk~\ref{ch:current system} wordt de werking van het huidige systeem in verband met de compilatie en bind van een programma toegelicht. 

In Hoofdstuk~\ref{ch:poc} wordt de uitvoering van de Proof Of Concept toegelicht en welke aanpassingen moesten gebeuren aan het product DBB van IBM. 

In Hoofdstuk~\ref{ch:conclusie}, tenslotte, wordt de conclusie gegeven en een antwoord geformuleerd op de onderzoeksvragen. Daarbij wordt ook een aanzet gegeven voor toekomstig onderzoek binnen dit domein.